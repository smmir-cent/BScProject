%% -!TEX root = AUTthesis.tex
% در این فایل، عنوان پایان‌نامه، مشخصات خود، متن تقدیمی‌، ستایش، سپاس‌گزاری و چکیده پایان‌نامه را به فارسی، وارد کنید.
% توجه داشته باشید که جدول حاوی مشخصات پروژه/پایان‌نامه/رساله و همچنین، مشخصات داخل آن، به طور خودکار، درج می‌شود.
%%%%%%%%%%%%%%%%%%%%%%%%%%%%%%%%%%%%
% دانشکده، آموزشکده و یا پژوهشکده  خود را وارد کنید
\faculty{دانشکده مهندسی کامپیوتر}
% گرایش و گروه آموزشی خود را وارد کنید
\department{}
% عنوان پایان‌نامه را وارد کنید
\fatitle{پیاده سازی سیستم پایش شبکه‌های کامپیوتری
\\[.75 cm]
}
% نام استاد(ان) راهنما را وارد کنید
\firstsupervisor{دکتر مسعود صبائی}
%\secondsupervisor{استاد راهنمای دوم}
% نام استاد(دان) مشاور را وارد کنید. چنانچه استاد مشاور ندارید، دستور پایین را غیرفعال کنید.
%\firstadvisor{دکتر رضا صفابخش}
%\secondadvisor{استاد مشاور دوم}
% نام نویسنده را وارد کنید
\name{سیدمهدی }
% نام خانوادگی نویسنده را وارد کنید
\surname{میرفندرسکی 9723093}
%%%%%%%%%%%%%%%%%%%%%%%%%%%%%%%%%%
\thesisdate{آبان 1400}

% چکیده پایان‌نامه را وارد کنید
\fa-abstract{
دوره کارآموزی در شرکت یافتار، در قالب پیاده سازی یک پروژه تحت عنوان روترهای توزیع شده طی شد. موضوع پروژه مربوط به شبکه‌های مخابراتی و بهبود عملکرد آن‌ها است.
\\
به کلاینتی که نقش روتر را ایفا خواهد کرد، یک ترافیک ورودی در قالب یک فایل استخراج شده از نرم افزار وایرشارک وارد می‌شود. حال به ازای هر بسته، بر اساس آی‌پی مقصد با استفاده از پروتکل دیامتر،درگاه بعدی کلاینت از سرور پرسیده می‌شود. بعد از آن سرور با اطلاعات موجود در پایگاه‌های داده، درگاهی که بسته از کلاینت باید خارج شود را به همراه آی‌پی مقصد بسته با پروتکل دیامتر به سمت کلاینت ارسال می‌کند. در تمام این مراحل اطلاعات بسته باید در کلاینت نیز ذخیره شود. همچنین باید دو سرور با داده‌های همگام وجود داشته باشد تا بتوانند در دو حالت فعال-فعال و یا فعال-منفعل برای کلاینت عمل کنند. هدف اصلی این پروژه، یافتن سریعترین و کاراترین پایگاه داده تحت شرایط بالا است. زبان مورد قبول پیاده سازی این پروژه سی و سی پلاس پلاس بود. همچنین نوع خاصی از سیستم عامل و نسخه برنامه‌ها باید مورد استفاده قرار می‌گرفت. بدین منظور پنج پایگاه داده در حافظه ردیس، تارانتول، هزلکست، اوراکل کوهیرنس و مونگودیبی را انتخاب کردیم و تست را انجام دادیم.\\نتیجه بر آن شد که پایگاه‌های داده ردیس و تارانتول با نتایج تقریبا برابر، بهتر از پایگاه‌های داده دیگر عمل کردند. سپس تصمیم گرفته شد که برای بحث همگام سازی بین دو سرور، برای پیاده سازی سه پایگاه داده برتر را انتخاب کنیم.
}


% کلمات کلیدی پایان‌نامه را وارد کنید
\keywords{شبکه‌های مخابراتی، روتر، دیامتر، پایگاه داده، ردیس، تارانتول، هزلکست، اوراکل کوهیرنس، مونگودیبی}



\AUTtitle
%%%%%%%%%%%%%%%%%%%%%%%%%%%%%%%%%%
\vspace*{7cm}
\thispagestyle{empty}
\begin{center}
\includegraphics[height=5cm,width=12cm]{besm}
\end{center}