\thispagestyle{empty}
\chapter{مقدمه}

پیشرفت روز افزون شبکه‌های کامپیوتری باعث شده است تا مدیریت آن نیز از اهمیت بالایی برخوردار باشد. اگر بخواهیم مروری اجمالی بر مفهوم مدیریت داشته باشیم، این گونه بیان می‌کنیم که مدیریت یک سیستم شامل پایش اجزا و جمع آوری داده، تحلیل داده‌ها و انجام اقدامات برای نزدیک شدن به هدف آن سیستم خواهد بود. به بیان دیگر مدیریت یک فرایند دائمی شامل پایش و نظارت، برنامه ریزی و اقدام است. مدیریت شبکه‌های کامپیوتری نیز به همین اجزا تقسیم بندی می‌شود.


هدف از انجام این پروژه توسعه یک ابزار مدیریتی به منظور پایش و نظارت شبکه‌های کامپیوتری است. به عبارت دیگر در این پروژه نرم افزاری توسعه داده می‌شود که به کمک این نرم افزار امکان پایش و نظارت تجهیزات قابل مدیریت شبکه، فراهم می‌گردد. این امکان از طریق یک رابط مدیریتی که اطلاعات پایش را از تجهیزات دریافت می‌کند و به نحوی قابل فهم برای مدیر شبکه نمایش می‌دهد، فراهم می‌گردد\cite{mauro2005essential}.

با پیاده سازی این ابزار، مدیر شبکه می‌تواند مشکلات شبکه را به موقع متوجه شود. همچنین می‌تواند برنامه ریزی کوتاه مدت و بلند مدت به منظور استفاده بهینه از منابع و جلوگیری از خرابی، انجام دهد.

پس از نگاه دقیق تری به مسئله پایش شبکه، حال ویژگی‌های مختلفی که ابزار پایش باید داشته باشد را بررسی می‌کنیم.

\section{ویژگی‌های ابزار پایش شبکه}

\subsection{انتخاب پروتکل مناسب و نحوه ارتباط}

در شبکه‌های کامپیوتری عناصری را می‌توان مدیریت کرد که بتوان اطلاعات مدیریتی را از آن دریافت کرد. به عبارتی عناصر تحت مدیریت، عناصری هستند که عامل مدیریتی بر روی آن‌ها راه اندازی شده باشد. دریافت این اطلاعات مدیریتی از عناصر باید از طریق یک پروتکل صورت بگیرد که به این نوع پروتکل‌ها، پروتکل مدیریتی گفته می‌شود.

اتخاذ یک پروتکل مدیریتی ایمن که مصرف پهنای باند حداقلی را داشته باشد بسیار مهم است.

\newpage

برای پایش یک شبکه و دستگاه‌های آن، اتخاذ یک پروتکل مدیریتی شبکه امن که مصرف پهنای باند حداقلی را داشته باشد بسیار برای ما مهم است. این پروتکل‌ها می‌توانند مانند \lr{SNMP}\LTRfootnote{\lr{Simple Network Management Protocol}}، استاندارد شده باشند. و یا حتی می‌توانند مانند \lr{CLI}\LTRfootnote{\lr{command-line interface}} غیر استاندارد باشند. اکثر عناصر تحت مدیریتی از پروتکل‌های \lr{SNMP} و \lr{CLI} پشتیبانی می‌کنند. همچنین دستگاه‌های ویندوزی از پروتکل \lr{WMI}\LTRfootnote{\lr{Windows Management Instrumentation}} نیز پشتیبانی می‌کنند.

پروتکل \lr{SNMP} یکی از پروتکل‌های لایه کاربرد برای مدیریت و پایش عناصر شبکه است. درواقع برای ارتباط با سیستم مدیریت شبکه تنها باید پیکربندی و فعال شود. به بیان دیگر نیازی به توسعه برنامه‌ای در سمت عناصر شبکه وجود ندارد\cite{hare2011simple}. این پروژه قصد استفاده از پروتکل \lr{SNMP} را دارد.




\subsection{کشف شبکه و دستگاه‌های مورد نیاز برای پایش}

در پایش شبکه، اولین قدم شناسایی عناصر قابل مدیریت و معیارهای عملکرد مرتبط با هر عنصر است. عناصری مانند رایانه‌های رومیزی و چاپگرها و مواردی از این دست برای ما حائز اهمیت نیستند و اساسا نیازی به پایش مداوم ندارند. از طرفی سرورها، روترها و سوئیچ‌ها وظایفی حیاتی را بر عهده دارند و نیاز به پایش مداوم دارند. ابزار پایش شبکه باید قادر باشد تا عناصر تحت مدیریت را شناسایی کند.


\subsection{دریافت اطلاعات و تنظیم پارامترهای مختلف}

دریافت اطلاعات از عناصر شبکه که قابل مدیریت باشند، به دو روش زیر صورت می‌گیرد:

\begin{itemize}
    \item ارسال درخواست سیستم مدیریت به عنصر تحت مدیریت و دریافت پاسخ از سمت عنصر تحت مدیریت
    \item ارسال نوتیفیکیشن\LTRfootnote{\lr{Notification}} بر اساس رخداد و یا رفتار غیر متعارف توسط عنصر تحت مدیریت و دریافت در سمت سیستم مدیریت
\end{itemize}

\newpage

در روش اول، درخواست‌های سیستم مدیریت شبکه به صورت دوره‌ای و متناوب انجام می‌شود. این دوره تناوب توسط مدیر شبکه باید قابل تنظیم باشد. تنظیم نادرست دوره تناوب می‌تواند منجر به استفاده غیر ضروری منابع و مصرف بی اندازه پهنای باند شود. 


اما در روش دوم نوتیفیکیشن‌ها بر اساس رخداد و یا رفتار غیر متعارف هستند. این رخداد و رفتار غیر متعارف باید توسط مدیر شبکه قابل تعریف باشد. این کار در سیستم مدیریت شبکه با تعریف حد آستانه‌هایی\LTRfootnote{\lr{Threshold}} در سمت عناصر تحت مدیریت صورت می‌گیرد، که این امر نیاز به تنظیم مدیر شبکه دارد. همچنین وجود حد آستانه‌های چند سطحی در شناسایی بهتر عناصر تحت مدیریت کمک می‌کند.


\subsection{هشدار فوری بر اساس نقض آستانه}


با استفاده از حد آستانه‌ها، هشدارهای پایش شبکه را می‌توان تا حدی قبل از رسیدن به شرایط بحرانی هوشمند کرد. به عنوان مثال می‌توان از ارسال ایمیل و یا نمایش در بستر وب بهره برد. بدین صورت که با عبور از هر حد آستانه، به مدیر شبکه از طریق ارسال ایمیل و نمایش بصری در بستر وب اطلاع رسانی شود.

