\chapter{مروری بر پروژه‌ها و سامانه‌های مشابه}


از زمانی که دستگاه‌ها در شبکه‌ها به هم متصل می‌شدند، نیاز به نوعی سیستم مدیریت و پایش شبکه وجود داشته است. در سال 1988 بود که \lr{SNMP} به استاندارد جدید تبدیل شد. هدف پروتکل \lr{SNMP} این است که زبانی برای انتقال اطلاعات مدیریتی شبکه بین دستگاه‌های مختلف به وجود آورد. امروزه اکثر دستگاه‌های شبکه می‌توانند \lr{SNMP} را به عنوان یک عامل راه اندازی کنند، به همین جهت اکثر نرم افزارهای پایش شبکه ارتباط از طریق \lr{SNMP} را در اولویت خود قرار می‌دهند.

\section{مروری بر پروژه‌های موجود}

امروزه ابزارهای متنوع و گوناگونی برای پایش شبکه توسعه داده شده‌اند. در ادامه به معرفی ابزارهای سولارویندز\LTRfootnote{\lr{SolarWinds}}، دیتاداگ\LTRfootnote{\lr{Datadog}} و زبیکس\LTRfootnote{\lr{Zabbix}} می‌پردازیم.



\subsection{سولارویندز}

این سیستم پایش شبکه از \lr{SNMP} برای بررسی وضعیت عناصر تحت مدیریت استفاده می‌کند. این ابزار قابلیت کشف عناصر شبکه را داراست به عبارتی دیگر دستگاه‌های موجود در شبکه که برای ما حائز اهمیت هستند را پیدا کرده و توپولوژی شبکه را ترسیم می‌کند. همچنین می‌توان یک توپولوژی مناسب برای کل زیرساخت شبکه طراحی کرد. به علاوه هشدارهای هوشمندی نیز دارد. اما در بحث نصب بر روی سیستم‌ عامل‌های مختلف محدودیت‌هایی وجود دارد، مثلا در بعضی توزیع‌های لینوکس بر پایه دبین\LTRfootnote{\lr{Debian}} قابل نصب نیست. برخی دیگر از ویژگی‌های سولارویندز عبارتند از:

\begin{itemize}
    \item قابلیت تجزیه و تحلیل مشکل: با فراهم آوردن دید کاملی از عملکرد زیرساخت شبکه، به هنگام وجود آمدن مشکل، پیدا کردن مبدا آن ساده خواهد بود.
    \item پایش عملکرد: این امکان را می‌دهد که بتوان بررسی کرد آیا اهداف عملکردی سرویس‌های مختلف برآورده شده‌اند یا خیر. این با پایش عملکرد در سطح برنامه‌های کاربردی، محقق می‌شود.
    \item سهولت استفاده: رابط کاربری کاربرپسند و ساده‌ای دارد.
\end{itemize}

\newpage

\subsection{دیتاداگ}

دیتاداگ یک ابزار پایش عملکرد شبکه است که مبتنی بر ابر\LTRfootnote{\lr{Software as a service}} است و این امکان را می‌دهد که ترافیک شبکه بین میزبان‌ها، کانتینرها و سرویس‌ها را در ابر تحلیل کنیم. برخی امکاناتی و تقاط قوتی که دیتاداگ دارد به شرح زیر می‌باشد:

\begin{itemize}
    \item این برنامه جریان ترافیک شبکه را می‌تواند بین میزبان‌ها، کانتینرها، شبکه‌های مختلف و حتی مفاهیم انتزاعی مانند سرویس‌ها و یا ماژول‌های مختلف نمایش می‌دهد. 
    \item در بحث عیب یابی، داده‌های ترافیک شبکه را با درنظر گرفتن برنامه‌های مربوطه، معیارهای دستگاه‌های مختلف و لاگ‌ها تحلیل می‌کند تا عیب‌یابی را در یک سیستم انجام دهد.
    \item به صورت بصری جریان ترافیک را نشان می‌دهد تا به شناسایی گلوگاه‌های ترافیکی کمک کند.
\end{itemize}






\subsection{زبیکس}

زبیکس یک ابزار پایش شبکه متن باز است که برای انواع عناصر تحت مدیریت خدمات ارائه می‌دهد. برخی امکاناتی که زبیکس در اختیار ما قرار می‌دهد به شرح زیر است\cite{olups2010zabbix}:

\begin{itemize}
    \item جمع آوری داده‌ها انعطاف پذیر است و با تغییر شبکه مشکلی پیش نخواهد آمد.
    \item توانایی شناسایی دستگاه‌های شبکه به طور خودکار را داراست.
    \item امکانات مختلفی برای هشدارها ارائه می‌دهد.
\end{itemize}

\newpage

\section{ویژگی‌های سامانه}

در نهایت بر اساس مطالعاتی که انجام شده و مطالب فصل قبل این سامانه باید قابلیت‌های زیر را داشته باشد:

\begin{itemize}
    \item کشف شبکه و ترسیم بصری آن برای ادراک بهتر توسط مدیر شبکه
    \item توانایی تنظیم پارامترها از جمله حد آستانه‌های عناصر تحت مدیریت بر اساس رفتار غیرمتعارف توسط مدیر شبکه و دوره تناوب جمع آوری اطلاعات از عناصر
    \item جمع آوری اطلاعات مدیریتی و پردازش آن‌ها جهت تولید اطلاعات قابل فهم توسط مدیر شبکه
    \item ایجاد یک رابط بصری مطلوب تحت وب و نمایش اطلاعات قابل فهم 
    \item فراهم آوردن حداقلی امنیت سیستم با استفاده از (\lr{SNMPv3} و پروتکل‌های دیگری که ایمن هستند)
    \item مقیاس پذیری سیستم جهت کارآمد بودن در هنگام افزایش وسعت شبکه
    \item فراهم آوردن یک مکانیزم هشدار با استفاده از حد آستانه‌های تعریف شده
\end{itemize}


به طور خلاصه هدف از انجام این پروژه توسعه یک ابزار پایش شبکه با ویژگی‌های پایه‌ای فوق است. به عبارتی هدف، افزودن یک امکان جدید به پروژه‌های موجود و پیاده سازی آن نیست، اما از زبان‌های برنامه نویسی و تکنولوژی‌های بروز در مقایسه با پروژه متن باز زبیکس استفاده خواهد شد. به عبارت دیگر این پروژه از ابتدا بدون استفاده از کدهای متن باز موجود توسعه داده خواهد شد. با انجام این پروژه دید خوبی نسبت به انجام یک پروژه صنعتی بدست خواهد آمد. همچنین چالش‌های مختلفی در ماژول‌ها و ارتباط بین آن‌ها برخورد می‌شود که باید رفع شوند.

