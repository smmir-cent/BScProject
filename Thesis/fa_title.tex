%% -!TEX root = AUTthesis.tex
% در این فایل، عنوان پایان‌نامه، مشخصات خود، متن تقدیمی‌، ستایش، سپاس‌گزاری و چکیده پایان‌نامه را به فارسی، وارد کنید.
% توجه داشته باشید که جدول حاوی مشخصات پروژه/پایان‌نامه/رساله و همچنین، مشخصات داخل آن، به طور خودکار، درج می‌شود.
%%%%%%%%%%%%%%%%%%%%%%%%%%%%%%%%%%%%
% دانشکده، آموزشکده و یا پژوهشکده  خود را وارد کنید
\faculty{دانشکده مهندسی کامپیوتر}
% گرایش و گروه آموزشی خود را وارد کنید
% \department{گرایش شبکه‌های کامپیوتری}
% عنوان پایان‌نامه را وارد کنید
\fatitle{
    \settextfont{B_Titr.ttf}
    پیاده‌سازی سیستم پایش شبکه‌های کامپیوتری}
% نام استاد(ان) راهنما را وارد کنید
\firstsupervisor{دکتر مسعود صبائی}
%\secondsupervisor{استاد راهنمای دوم}
% نام استاد(دان) مشاور را وارد کنید. چنانچه استاد مشاور ندارید، دستور پایین را غیرفعال کنید.
% \firstadvisor{دکتر سیاوش خرسندی}
%\secondadvisor{استاد مشاور دوم}
% نام نویسنده را وارد کنید
\name{سیدمهدی }
% نام خانوادگی نویسنده را وارد کنید
\surname{میرفندرسکی}
%%%%%%%%%%%%%%%%%%%%%%%%%%%%%%%%%%
\thesisdate{شهریور 1401}

% چکیده پایان‌نامه را وارد کنید
\fa-abstract{
    پیشرفت روز افزون شبکه‌هاي کامپیوتري باعث شده است تا مدیریت آن از اهمیت بالایی برخوردار باشد. به طور کلی مدیریت یک سیستم شامل پایش اجزا و جمع‌آوري اطلاعات، تحلیل اطلاعات و انجام اقدامات براي نزدیک شدن به هدف آن سیستم خواهد بود. به بیانی دیگر مدیریت یک فرایند دائمی شامل پایش و نظارت، برنامه‌ریزي و اقدام است. مدیریت شبکه‌هاي کامپیوتري نیز شامل این فعالیت‌ها خواهد بود.
    \\
    پایش اجزا و جمع‌آوری اطلاعات از اساسی‌ترین فرایندهای مدیریت است. با به‌کارگیری این فرایند می‌توان اقدامات کوتاه‌مدت یا بلندمدت براي برنامه‌ریزي انجام داد. همچنین می‌توان توپولوژی شبکه را به همراه نوع دستگاه‌ها با جمع‌آوری اطلاعات بدست آورد (کشف شبکه). درنهایت این اقدامات کوتاه‌مدت و بلندمدت برگرفته از پایش اجزا، شبکه را به سمت شبکه‌ای با کارایی بالا هدایت می‌نماید. درواقع سامانه پایش شبکه این امکان را فراهم می‌آورد که بتوان عملکرد هر عنصر تحت مدیریت را پایش و هشدارهای مدیریتی از شبکه را جمع‌آوری کرد.
    \\
    هدف از انجام این پروژه توسعه یک ابزار مدیریتی به منظور پایش و نظارت شبکه‌هاي کامپیوتري است. به عبارت دیگر در این پروژه، نرم افزاري توسعه داده می‌شود که به کمک آن امکان پایش و نظارت تجهیزات قابل مدیریت شبکه، فراهم گردد. این امکان از طریق یک واسط مدیریتی که اطلاعات پایش را از تجهیزات دریافت می‌کند و به نحوي قابل فهم براي مدیر شبکه نمایش می دهد، فراهم می‌گردد. دریافت اطلاعات مدیریتی از عناصر از طریق پروتکل‌های مدیریتی صورت می‌گیرد. اتخاذ یک پروتکل مدیریتی ایمن که مصرف پهناي باند حداقلی را داشته باشد بسیار مهم است. پروتکل \lr{SNMP} یکی از پروتکل‌های لایه کاربرد برای مدیریت و پایش عملکرد عناصر شبکه است که در این سامانه مورد استفاده قرار می‌گیرد. پس از دریافت اطلاعات در سامانه، آن‌ها پردازش شده و در پایگاه‌های داده ذخیره خواهند شد. همچنین واسط کاربری تحت وبی نیز با واکشی این اطلاعات، آن‌ها را به کاربر نمایش می‌دهد.}


% کلمات کلیدی پایان‌نامه را وارد کنید
\keywords{مدیریت شبکه، پایش شبکه، پروتکل \lr{SNMP}، کشف شبکه }



\AUTtitle
%%%%%%%%%%%%%%%%%%%%%%%%%%%%%%%%%%
\vspace*{7cm}
\thispagestyle{empty}
\begin{center}
\includegraphics[height=5cm,width=12cm]{besm}
\end{center}