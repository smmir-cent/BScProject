%% -!TEX root = AUTthesis.tex
% در این فایل، عنوان پایان‌نامه، مشخصات خود، متن تقدیمی‌، ستایش، سپاس‌گزاری و چکیده پایان‌نامه را به فارسی، وارد کنید.
% توجه داشته باشید که جدول حاوی مشخصات پروژه/پایان‌نامه/رساله و همچنین، مشخصات داخل آن، به طور خودکار، درج می‌شود.
%%%%%%%%%%%%%%%%%%%%%%%%%%%%%%%%%%%%
% دانشکده، آموزشکده و یا پژوهشکده  خود را وارد کنید
\faculty{دانشکده مهندسی کامپیوتر}
% گرایش و گروه آموزشی خود را وارد کنید
% \department{گرایش شبکه‌های کامپیوتری}
% عنوان پایان‌نامه را وارد کنید
\fatitle{
    \settextfont{B_Titr.ttf}
    پیاده‌سازی سیستم پایش شبکه‌های کامپیوتری}
% نام استاد(ان) راهنما را وارد کنید
\firstsupervisor{دکتر مسعود صبائی}
%\secondsupervisor{استاد راهنمای دوم}
% نام استاد(دان) مشاور را وارد کنید. چنانچه استاد مشاور ندارید، دستور پایین را غیرفعال کنید.
% \firstadvisor{دکتر سیاوش خرسندی}
%\secondadvisor{استاد مشاور دوم}
% نام نویسنده را وارد کنید
\name{سیدمهدی }
% نام خانوادگی نویسنده را وارد کنید
\surname{میرفندرسکی}
%%%%%%%%%%%%%%%%%%%%%%%%%%%%%%%%%%
\thesisdate{شهریور 1401}

% چکیده پایان‌نامه را وارد کنید
% پایش اجزا و جمع‌آوری اطلاعات از اساسی‌ترین فرایندهای مدیریت است. با به‌کارگیری این فرایند می‌توان اقدامات کوتاه‌مدت یا بلندمدت براي برنامه‌ریزي انجام داد. همچنین می‌توان توپولوژی شبکه را به همراه نوع دستگاه‌ها با جمع‌آوری اطلاعات بدست آورد (کشف شبکه). درنهایت این اقدامات کوتاه‌مدت و بلندمدت برگرفته از پایش اجزا، شبکه را به سمت شبکه‌ای با کارایی بالا هدایت می‌نماید. درواقع سامانه پایش شبکه این امکان را فراهم می‌آورد که بتوان عملکرد هر عنصر تحت مدیریت را پایش و هشدارهای مدیریتی از شبکه را جمع‌آوری کرد.
\fa-abstract{
    پیشرفت روز افزون شبکه‌هاي کامپیوتري باعث شده است تا مدیریت آن از اهمیت بالایی برخوردار باشد. به طور کلی مدیریت یک سیستم شامل پایش اجزا، تحلیل اطلاعات و انجام اقدامات براي نزدیک شدن به هدف آن سیستم خواهد بود. به بیانی دیگر مدیریت و به طور ویژه مدیریت شبکه‌های کامپیوتری یک فرایند دائمی شامل پایش، پردازش، برنامه‌ریزي و اقدام است. هدف از انجام این پروژه توسعه ابزاری جهت پایش شبکه‌هاي کامپیوتري بود. با توسعه این سامانه، اطلاعات مدیریتی شامل اطلاعات ترافیکی، اطلاعات پیکربندی، هشدارها و ... جمع‌آوری شده و به کاربر نمایش داده می‌شود. تا با این امکان، کاربر بتواند از طریق این اطلاعات، برنامه‌ریزی و اقدام کند. این سامانه از طریق یک واسط مدیریتی اطلاعات پایش را از اجزا با پروتکل \lr{SNMP} دریافت می‌کند و به نحوي قابل فهم براي مدیر شبکه نمایش می‌دهد. اطلاعات مدیریتی پس از دریافت در سامانه، پردازش می‌شوند و در انتها در پایگاه‌های داده ذخیره خواهند شد. همچنین واسط کاربری تحت وبی نیز با واکشی این اطلاعات مدیریتی، آن‌ها را به کاربر نمایش می‌دهد. امکان دیگری که به این سامانه اضافه شد تا برنامه‌ریزی و اقدام را برای مدیریت تسهیل نماید، کشف شبکه بود. با این قابلیت کاربر می‌تواند یک دید کلی از شبکه نیز بدست آورد. درانتهای پروژه نیز نیازمندی‌های مختلف سامانه از جنبه کارکردی و غیرکارکردی آن مورد بررسی و تست قرار گرفتند.
    }


% کلمات کلیدی پایان‌نامه را وارد کنید
\keywords{مدیریت شبکه، پایش شبکه، پروتکل \lr{SNMP}، کشف شبکه }



\AUTtitle
%%%%%%%%%%%%%%%%%%%%%%%%%%%%%%%%%%
\vspace*{7cm}
\thispagestyle{empty}
\begin{center}
\includegraphics[height=5cm,width=12cm]{besm}
\end{center}