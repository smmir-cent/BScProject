\chapter{مقدمه}

پیشرفت روز افزون شبکه‌هاي کامپیوتري باعث شده است تا مدیریت آن نیز از اهمیت بالایی برخوردار باشد. اگر بخواهیم مروري اجمالی بر مفهوم مدیریت داشته باشیم، این گونه بیان می‌کنیم که مدیریت یک سیستم شامل پایش\LTRfootnote{\lr{Monitoring}} اجزا و جمع آوري داده، تحلیل داده‌ها و انجام اقدامات براي نزدیک شدن به هدف آن سیستم خواهد بود. به بیان دیگر مدیریت یک فرایند دائمی شامل پایش و نظارت، برنامه ریزي و اقدام است. مدیریت شبکه‌هاي کامپیوتري نیز به همین اجزا تقسیم بندي می‌شود.

هدف از انجام این پروژه توسعه یک ابزار مدیریتی به منظور پایش و نظارت شبکه‌هاي کامپیوتري است. به عبارت دیگر در این پروژه، نرم افزاري توسعه داده می‌شود که به کمک این نرم افزار امکان پایش و نظارت تجهیزات قابل مدیریت شبکه، فراهم گردد. این امکان از طریق یک رابط مدیریتی که اطلاعات پایش را از تجهیزات دریافت می‌کند و به نحوي قابل فهم براي مدیر شبکه نمایش می‌دهد، فراهم می‌گردد.

با پیاده سازي این ابزار، مدیر شبکه می‌تواند مشکلات شبکه را به موقع متوجه شود. همچنین می‌تواند برنامه‌ریزي کوتاه مدت و بلند مدت به منظور استفاده بهینه از منابع و جلوگیري از خرابی، انجام دهد. پس از نگاه دقیق تري به مسئله پایش شبکه، حال ویژگی‌هاي مختلفی که این سامانه باید داشته باشد را بیان می‌کنیم:

\begin{itemize}
    \item انتخاب پروتکل مناسب و نحوه ارتباط: دریافت اطلاعات مدیریتی از عناصر باید از طریق یک پروتکل صورت بگیرد که به این نوع پروتکل‌ها، پروتکل مدیریتی گفته می‌شود. اتخاذ یک پروتکل مدیریتی ایمن که مصرف پهناي باند حداقلی را داشته باشد بسیار مهم است. این پروتکل‌ها می توانند مانند\lr{SNMP}\LTRfootnote{\lr{Simple Network Management Protocol}} ،استاندارد شده باشند و یا حتی می‌توانند مانند \lr{CLI}\LTRfootnote{\lr{Command-line interface}} غیر استاندارد باشند. اکثر عناصر تحت مدیریتی از پروتکل‌هاي \lr{SNMP} و  \lr{CLI} پشتیبانی می‌کنند. همچنین دستگاه‌هاي ویندوزي از پروتکل \lr{WMI}\LTRfootnote{\lr{Windows Management Instrumentation}} نیز پشتیبانی می‌کنند. پروتکل  \lr{SNMP}یکی از پروتکل‌هاي لایه کاربرد \LTRfootnote{\lr{Application layer}} براي مدیریت و پایش عناصر شبکه است. درواقع براي ارتباط با سیستم مدیریت شبکه تنها باید پیکربندي و فعال شود. به بیان دیگر نیازي به توسعه برنامه‌اي در سمت عناصر شبکه وجود ندارد. این پروژه قصد استفاده از پروتکل \lr{SNMP} را دارد.
\newpage
    \item کشف شبکه\LTRfootnote{\lr{Network Discovery}} و دستگاه‌هاي مورد نیاز براي پایش: در پایش شبکه، اولین قدم شناسایی عناصر قابل مدیریت و معیارهاي عملکرد مرتبط با هر عنصر است. عناصري مانند رایانه‌هاي رومیزي و چاپگرها و مواردي از این دست براي ما حائز اهمیت نیستند و اساسا نیازي به پایش مداوم ندارند. از طرفی سرورها\LTRfootnote{\lr{Servers}}، روترها\LTRfootnote{\lr{Routers}} و سوئیچ‌ها\LTRfootnote{\lr{Switches}} وظایفی حیاتی را بر عهده دارند و نیاز به پایش مداوم دارند. ابزار پایش شبکه باید قادر باشد تا عناصر تحت مدیریت را شناسایی کند.
    \item دریافت اطلاعات و تنظیم پارامترهاي مختلف: دریافت اطلاعات از عناصر شبکه که قابل مدیریت باشند، به دو صورت انجام می‌پذیرد:
    \begin{enumerate}
        \item ارسال درخواست سیستم مدیریت به عنصر تحت مدیریت و دریافت پاسخ از سمت عنصر تحت مدیریت
        \item ارسال نوتیفیکیشن\LTRfootnote{\lr{Notification}} بر اساس رخداد و یا رفتار غیر متعارف توسط عنصر تحت مدیریت و دریافت در سمت سیستم مدیریت
    \end{enumerate}
\end{itemize}


متدولوژی توسعه این سامانه بدین صورت است که، ابتدا نیازمندی‌ها جمع آوری می‌شوند، بعد از آن تکنولوژی‌های توسعه سامانه انتخاب می‌شوند. حال که تکنولوژی سامانه تعیین شد، معماری سامانه طراحی می‌شود. بعد از طراحی معماری، پیاده سازی سامانه صورت می‌گیرد. درنهایت نیز سامانه بر اساس تحلیل نیازمندی‌ها تست می‌شود. 

همچنین مراحل گفته شده برای توسعه این سامانه در ادامه این پایان نامه آورده شده است. ابتدا در فصل دوم مروری بر سامانه‌های مشابه خواهیم داشت. سپس در فصل سوم به ترتیب به جمع آوری نیازمندی‌ها و تحلیل آن‌ها، انتخاب تکنولوژی و طراحی معماری می‌پردازیم. در فصل چهارم شیوه پیاده سازی این سامانه بیان می‌شود و در فصل پنجم تست‌های صورت گرفته مطرح خواهند شد. در فصل آخر نیز از نتیجه‌گیری و کارهای آینده صحبت خواهیم کرد.

