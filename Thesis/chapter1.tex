\chapter{مقدمه}

پیشرفت روز افزون شبکه‌های کامپیوتری باعث شده است تا مدیریت آن نیز از اهمیت بالایی برخوردار باشد. اگر بخواهیم مروری اجمالی بر مفهوم مدیریت داشته باشیم، این گونه بیان می‌کنیم که مدیریت یک سیستم شامل پایش\LTRfootnote{\lr{Monitoring}} اجزا و جمع‌آوری اطلاعات، تحلیل اطلاعات و انجام اقدامات برای نزدیک شدن به هدف آن سیستم خواهد بود. به بیان دیگر مدیریت یک فرایند دائمی شامل پایش و نظارت، برنامه‌ریزی و اقدام است. مدیریت شبکه‌های کامپیوتری نیز شامل این فعالیت‌ها خواهد بود.

هدف از انجام این پروژه توسعه یک ابزار مدیریتی به منظور پایش و نظارت شبکه‌های کامپیوتری است. به عبارت دیگر در این پروژه، نرم‌افزاری توسعه داده می‌شود که به کمک آن امکان پایش و نظارت تجهیزات قابل مدیریت شبکه، فراهم گردد. این امکان از طریق یک واسط مدیریتی که اطلاعات پایش را از تجهیزات دریافت می‌کند و به نحوی قابل فهم برای مدیر شبکه نمایش می‌دهد، فراهم می‌گردد\cite{mauro2005essential}.

با پیاده‌سازی این ابزار، مدیر شبکه می‌تواند مشکلات شبکه را به موقع متوجه شود. همچنین می‌تواند برنامه‌ریزی کوتاه‌مدت و بلندمدت به منظور استفاده بهینه از منابع و جلوگیری از خرابی، انجام دهد. پس از نگاه دقیق تری به مسئله پایش شبکه، حال ویژگی‌های مختلفی که این سامانه باید داشته باشد در ادامه بیان می‌شود:

\begin{itemize}
    \item انتخاب پروتکل مناسب و نحوه ارتباط: دریافت اطلاعات مدیریتی از عناصر باید از طریق یک پروتکل صورت بگیرد که به این نوع پروتکل‌ها، پروتکل مدیریتی گفته می‌شود. اتخاذ یک پروتکل مدیریتی ایمن که مصرف پهنای باند حداقلی را داشته باشد بسیار مهم است. این پروتکل‌ها می‌توانند مانند\lr{SNMP}\LTRfootnote{\lr{Simple Network Management Protocol}}، استاندارد شده باشند و یا حتی می‌توانند مانند \lr{CLI}\LTRfootnote{\lr{Command-line interface}} غیر استاندارد باشند. اکثر عناصر تحت مدیریتی از پروتکل‌های \lr{SNMP} و  \lr{CLI} پشتیبانی می‌کنند. همچنین دستگاه‌های ویندوزی از پروتکل \lr{WMI}\LTRfootnote{\lr{Windows Management Instrumentation}} نیز پشتیبانی می‌کنند. پروتکل  \lr{SNMP} یکی از پروتکل‌های لایه کاربرد\LTRfootnote{\lr{Application layer}} برای مدیریت و پایش عملکرد عناصر شبکه است. درواقع برای ارتباط با سیستم مدیریت شبکه تنها باید پیکربندی و فعال شود. به بیان دیگر نیازی به توسعه برنامه‌ای در سمت عناصر شبکه وجود ندارد\cite{hare2011simple}. این پروژه قصد استفاده از پروتکل \lr{SNMP} را دارد.
\newpage
    \item کشف شبکه\LTRfootnote{\lr{Network Discovery}} و دستگاه‌های مورد نیاز برای پایش: در پایش شبکه، اولین قدم شناسایی عناصر قابل مدیریت و معیارهای عملکرد مرتبط با هر عنصر است. عناصری مانند رایانه‌های رومیزی و چاپگرها و مواردی از این دست برای ما حائز اهمیت نیستند و اساسا نیازی به پایش مداوم ندارند. از طرفی سرورها\LTRfootnote{\lr{Servers}}، مسیریاب‌ها\LTRfootnote{\lr{Routers}} و سوئیچ‌ها\LTRfootnote{\lr{Switches}} وظایفی حیاتی را بر عهده دارند و نیاز به پایش مداوم دارند. ابزار پایش شبکه باید قادر باشد تا عناصر تحت مدیریت را شناسایی کند.
    \item دریافت اطلاعات و تنظیم پارامترهای مختلف: دریافت اطلاعات از عناصر شبکه که قابل مدیریت باشند، به دو صورت انجام می‌پذیرد:
    \begin{enumerate}
        \item ارسال درخواست سیستم مدیریت به عنصر تحت مدیریت و دریافت پاسخ از سمت عنصر تحت مدیریت
        \item ارسال اعلان\LTRfootnote{\lr{Notification}} بر اساس رخداد و یا رفتار غیر متعارف توسط عنصر تحت مدیریت و دریافت در سمت سیستم مدیریت
    \end{enumerate}
\end{itemize}


روش\LTRfootnote{\lr{Methodology}} توسعه این سامانه بدین صورت است که، ابتدا نیازمندی‌ها جمع‌آوری می‌شوند، بعد از آن فناوری‌های\LTRfootnote{\lr{Technology}} توسعه سامانه انتخاب می‌شوند. حال که فناوری سامانه تعیین شد، معماری سامانه طراحی می‌شود. بعد از طراحی معماری، پیاده‌سازی سامانه صورت می‌گیرد. درنهایت نیز سامانه بر اساس تحلیل نیازمندی‌ها تست می‌شود. 

همچنین مراحل گفته شده برای توسعه این سامانه در ادامه این پایان‌نامه آورده شده است. ابتدا فصل دوم مروری بر سامانه‌های مشابه خواهد داشت. سپس فصل سوم به ترتیب به جمع‌آوری نیازمندی‌ها و تحلیل آن‌ها، انتخاب فناوری و طراحی معماری می‌پردازد. در فصل چهارم شیوه پیاده‌سازی این سامانه بیان می‌شود و در فصل پنجم تست‌های صورت گرفته مطرح خواهند شد. در فصل آخر نیز از نتیجه‌گیری و کارهای آینده مطالبی بیان خواهد شد.

