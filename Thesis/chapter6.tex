\chapter{جمع‌بندی}
%%%%%%%%%%%%%%%%%%%%%%%%%%%%%%%%%%%%%%%%%%%
مقدمه فصل

\section{خلاصه}

در این پروژه هدف، پیاده‌سازی یک سامانه برای پایش شبکه‌های کامپیوتری بوده است. برای تحقق این هدف، ابتدا با مروری بر روی سامانه‌های موجود، زیرمجموعه‌هایی از ویژگی‌ها استخراج شدند. پس از آن با تحلیل نیازمندی‌ها، جدول‌های نیازمندی‌های کارکردی و غیرکارکردی استخراج شدند. پس از استخراج نیازمندی‌ها جهت طراحی معماری، فناوری‌های پیاده‌سازی بررسی و برای هر بخش یک فناوری انتخاب شد. باتوجه به معماری خود فناوری‌های انتخاب شده، معماری کل سامانه طراحی شد. سپس پیاده‌سازی بر اساس ماژول‌های مختلف جهت توسعه نرم‌افزار انجام شد. درنهایت نیز تست سامانه موجود بر اساس نیازمندی‌های تحلیل شده، انجام شد.


\section{کاربرد‌ها}

این پروژه کاربرد‌های بسیار در شبکه‌های مختلف دارد. مدیر شبکه با به کارگیری این سامانه می‌تواند عملکرد یک سیستم یا کل شبکه را پایش کند و اقدامات کوتاه‌مدت یا بلندمدت برای برنامه‌ریزی و انجام دهد. اقدامات کوتاه‌مدت شامل تغییر تنظیمات، تغییر مسیرهای مسیریاب‌ها و ... هستند. همچنین اقدامات بلندمدت شامل تهیه منابع مختلف، بهبود سرویس‌های موجود و ... هستند. همچنین هشدارهای تولید شده در سطح شبکه به راحتی قابل مشاهده خواهند بود. کاربرد دیگر این سامانه در بخش کشف شبکه خواهد بود. مدیر شبکه می‌تواند با اسکن کل شبکه، توپولوژی را به همراه نوع ماشین‌های مختلف مشاهده کند تا در نهایت شبکه را به سمت شبکه‌ای با کارایی بالا هدایت نماید.

\newpage

\section{کار‌های آتی}

در ماژول پایش شبکه مفهومی تحت عنوان نرخ نمونه‌برداری وجود دارد. این مفهوم بدین معناست که مقادیر عملکردی عناصر تحت مدیریت، با چه تناوبی نمونه‌برداری شوند. در حال حاضر این مقدار توسط مدیر شبکه باید تنظیم شود. اما می‌توان این مقادیر را با استفاده از الگوریتم‌های هوش مصنوعی و تحت عنوان پایش هوشمند متناسب با شبکه و مقادیر تنظیم شود. این کار سه مزیت اصلی خواهد داشت:

\begin{itemize}
    \item عدم درگیری مدیر شبکه با مقادیر نرخ نمونه‌برداری
    \item بهینه‌سازی مقدار ترافیک مدیریتی شبکه
    \item دقت بالاتر برای رسم نمودارها
\end{itemize}


از کارهای دیگر می‌توان به کاهش حجم اطلاعات در سامانه اشاره کرد. در بلند مدت حجم این اطلاعات بسیار زیاد خواهد شد. دو اقدام اصلی می‌توان جهت این مشکل انجام داد:

\begin{itemize}
    \item تغییر ساختار اطلاعات به نحوی بهینه
    \item پردازش دوره‌ای اطلاعات و نگهداری خلاصه‌ای از آن‌ها از جمله مقادیر آماری(میانگین، انحراف معیار و ...)
\end{itemize}