\chapter{مروری بر سامانه‌های مشابه}

از زمانی که دستگاه‌ها در شبکه‌ها به هم متصل می‌شدند، نیاز به نوعی سیستم مدیریت و پایش شبکه وجود داشته است. در سال 1988 م. بود که \lr{SNMP} به استاندارد جدید تبدیل شد. هدف پروتکل \lr{SNMP} این است که زبانی برای انتقال اطلاعات مدیریتی شبکه بین دستگاه‌های مختلف به وجود آورد. امروزه اکثر دستگاه‌های شبکه می‌توانند \lr{SNMP} را به عنوان یک عامل راه‌اندازی کنند، به همین جهت اکثر نرم‌افزارهای پایش شبکه ارتباط از طریق \lr{SNMP} را در اولویت خود قرار می‌دهند\cite{noauthor_snmp_nodate}.
از سال‌ها پیش با طراحی اینگونه سامانه‌ها کارهای ارزشمندی انجام شده است. این فصل به معرفی اجمالی بعضی از این سامانه‌ها می‌پردازد.

\section{مروری بر سامانه‌های موجود}

امروزه ابزارهای متنوع و گوناگونی برای پایش شبکه توسعه داده شده‌اند. در ادامه ابزارهای سولارویندز\LTRfootnote{\lr{SolarWinds}}، دیتاداگ\LTRfootnote{\lr{Datadog}} و زبیکس\LTRfootnote{\lr{Zabbix}} معرفی می‌شوند.



\subsection{سولارویندز}

این سیستم پایش شبکه از \lr{SNMP} برای بررسی وضعیت عناصر تحت مدیریت استفاده می‌کند. این ابزار قابلیت کشف عناصر شبکه را داراست. به عبارتی دیگر دستگاه‌های موجود در شبکه که برای ما حائز اهمیت هستند را پیدا کرده و توپولوژی شبکه را ترسیم می‌کند. همچنین می‌توان یک توپولوژی مناسب برای کل زیرساخت شبکه طراحی کرد. به علاوه هشدارهای هوشمندی نیز دارد\cite{noauthor_solarwinds_nodate}. اما در بحث نصب بر روی سیستم‌ عامل‌های مختلف محدودیت‌هایی وجود دارد، مثلا در بعضی توزیع‌های لینوکس\LTRfootnote{\lr{Linux}} بر پایه دبین\LTRfootnote{\lr{Debian}} قابل نصب نیست.

\newpage


برخی دیگر از ویژگی‌های سولارویندز عبارتند از:

\begin{itemize}
    \item قابلیت تجزیه و تحلیل مشکل: با فراهم آوردن دید کاملی از عملکرد زیرساخت شبکه، به هنگام وجود آمدن مشکل، پیدا کردن مبدا آن ساده خواهد بود.
    \item پایش عملکرد: این امکان را می‌دهد که بتوان بررسی کرد آیا اهداف عملکردی سرویس‌های مختلف برآورده شده‌اند یا خیر. این با پایش عملکرد در سطح برنامه‌های کاربردی، محقق می‌شود.
    \item سهولت استفاده: واسط کاربری کاربرپسند و ساده‌ای دارد.
\end{itemize}



همچنین از معایب این سامانه می‌توان به موارد زیر اشاره کرد:



\begin{itemize}
    \item در صورت عدم سفارشی‌سازی هشدارهای دریافتی، مقدار زیادی هشدار دریافت می‌شود و چون مقدار آن‌ها زیاد است، توسط کاربر نادیده گرفته می‌شوند.
    \item برای بعضی کاربران این سامانه، واسط کاربری گاهی اوقات می‌تواند گیج‌کننده باشد.
    \item برای بعضی کاربردها هشدارها مبهم می‌باشد و نیاز کاربران برطرف نمی‌شود.
    \item برای استفاده از این سامانه به طور کامل باید هزینه پرداخت شود.
\end{itemize}


\subsection{دیتاداگ}

دیتاداگ یک ابزار پایش عملکرد شبکه است که مبتنی بر ابر\LTRfootnote{\lr{Software as a service}} است و این امکان را می‌دهد که ترافیک شبکه بین میزبان‌ها\LTRfootnote{\lr{Hosts}}، کانتینرها\LTRfootnote{\lr{Containers}} و سرویس‌ها را در ابر تحلیل کنیم\cite{noauthor_what_nodate}.

\newpage
برخی امکانات و نقاط قوتی که دیتاداگ دارد به شرح زیر می‌باشد:

\begin{itemize}
    \item این برنامه جریان ترافیک شبکه را می‌تواند بین میزبان‌ها، کانتینرها، شبکه‌های مختلف و حتی مفاهیم انتزاعی مانند سرویس‌ها و یا ماژول‌های مختلف نمایش می‌دهد. 
    \item داده‌های ترافیک شبکه را با درنظر گرفتن برنامه‌های مربوطه، معیارهای دستگاه‌های مختلف و لاگ‌ها تحلیل می‌کند تا عیب‌یابی را در یک سیستم انجام دهد.
    \item به صورت بصری جریان ترافیک را نشان می‌دهد تا به شناسایی گلوگاه‌های ترافیکی کمک کند.
\end{itemize}

همچنین از معایب این سامانه می‌توان به موارد زیر اشاره کرد:


\begin{itemize}
    \item استفاده از این سامانه برای کاربران جدید شاید بسیار سخت باشد ازطرفی مستندسازی شیوا و فصیحی ندارد، به همین دلیل باید زمانی صرف ارتباط با پشتیبانی شود.
    \item همچنین گزارش شده است که گاهی اوقات درک نمودارها بسیار دشوار است و استفاده از آن به دانش فنی نیاز دارد.
    \item همچنین مانند سولارویندز برای استفاده از آن، باید هزینه پرداخت شود.
\end{itemize}


\subsection{زبیکس}

زبیکس یک ابزار پایش شبکه متن‌باز\LTRfootnote{\lr{Open-Source}} است که برای انواع عناصر تحت مدیریت خدمات ارائه می‌دهد. برخی امکاناتی که زبیکس در اختیار ما قرار می‌دهد به شرح زیر است\cite{olups2010zabbix}:

\begin{itemize}
    \item جمع‌آوری اطلاعات انعطاف پذیر است و با تغییر شبکه مشکلی پیش نخواهد آمد.
    \item توانایی شناسایی دستگاه‌های شبکه به طور خودکار را داراست.
    \item امکانات مختلفی برای هشدارها ارائه می‌دهد.
\end{itemize}
\newpage
همچنین از معایب این سامانه می‌توان به موارد زیر اشاره کرد:

\begin{itemize}
    \item برخی از خطاها اطلاعات کافی برای عیب‌یابی ارائه نمی‌دهند.
    \item برخی مستندات این سامانه باید بروز شوند.
    \item انعطاف پذیری دیرهنگام با سرویس‌های ابری مانند \lr{AWS}\LTRfootnote{\lr{Amazon Web Services}}
\end{itemize}

\section{خلاصه}

در این فصل بر روی سامانه‌های مشابه در حوزه پایش شبکه، مروری انجام شد. ابتدا تاریخچه‌ و اهمیت پروتکل \lr{SNMP} ارائه شد. بعد از آن به ترتیب به سه ابزار سولارویندز، دیتاداگ و زبیکس به صورت خلاصه پرداخته شد. از نقاط ضعف سامانه سولارویندز به قابل نصب نبودن بر روی برخی توزیع‌های لینوکس بر پایه دبین، هزینه‌بر بودن آن و تعداد هشدار بالای آن اشاره شد. البته نقاط قوت خوبی ازجمله قابلیت تجزیه و تحلیل مشکل و سهولت استفاده نیز ذکر شد. بعد از مطرح کردن سولارویندز، دیتاداگ که یک ابزار پایش مبتنی بر ابر بود معرفی شد. از نقاط ضعف این سامانه به کیفیت پایین مستندات آن، هزینه‌بر بودن آن و نیاز به دانش فنی برای درک نمودارهای آن اشاره شد. همچنین برای نقاط قوت به نشان دادن جریان ترافیک به صورت بصری بین میزبان‌ها، کانتینرها، شبکه‌های مختلف و سرویس‌ها اشاره شد. درنهایت نیز به سامانه زبیکس که نقطه قوت متمایز آن رایگان و متن‌باز بودن آن بود، پرداخته شد. البته معایبی مثل عدم ارائه اطلاعات کافی خطاها برای آن نیز ذکر شد.
